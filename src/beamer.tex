\documentclass{beamer}
\usepackage[spanish]{babel}
\usepackage[utf8]{inputenc}
\usepackage{graphicx}

\newtheorem{definicion}{Definicion}

%%%%%%%%%%%%%%%%%%%%%%%%%%%%%%%%%%%%%%%%%%%%%%%%%%%%%%%%%%%%%%%%%%%%%%%%%%%%%%%

\title[Beamer]{Beamer}
\author[cpcroberto]{Roberto Carlos Palenzuela Criado}
\date[\today]{\today}

%%%%%%%%%%%%%%%%%%%%%%%%%%%%%%%%%%%%%%%%%%%%%%%%%%%%%%%%%%%%%%%%%%%%%%%%%%%%%%%

\usetheme{Madrid}
\usecolortheme[RGB={122,59,122}]{structure}
\setbeamercovered{transparent}

%%%%%%%%%%%%%%%%%%%%%%%%%%%%%%%%%%%%%%%%%%%%%%%%%%%%%%%%%%%%%%%%%%%%%%%%%%%%%%%

\begin{document}
  
%%%%%%%%%%%%%%%%%%%%%%%%%%%%%%%%%%%%%%%%%%%%%%%%%%%%%%%%%%%%%%%%%%%%%%%%%%%%%%%

\begin{frame}
  \titlepage
\end{frame}

%%%%%%%%%%%%%%%%%%%%%%%%%%%%%%%%%%%%%%%%%%%%%%%%%%%%%%%%%%%%%%%%%%%%%%%%%%%%%%%

\section{Introducción}

%%%%%%%%%%%%%%%%%%%%%%%%%%%%%%%%%%%%%%%%%%%%%%%%%%%%%%%%%%%%%%%%%%%%%%%%%%%%%%

\begin{frame}
  \frametitle{¿Que es Beamer?}
  \begin{definicion}
    Beamer es una clase de \LaTeX{} para la creación de presentaciones.
  \end{definicion}
\end{frame}

%%%%%%%%%%%%%%%%%%%%%%%%%%%%%%%%%%%%%%%%%%%%%%%%%%%%%%%%%%%%%%%%%%%%%%%%%%%%%%%%%%%%%%%%%

\begin{frame}
  \frametitle{¿Que es Beamer?}
  \begin{block}{Características}
    \begin{itemize}[<+->]
      \item {\large Todas las ventajas heredadas de \LaTeX{}.} \\
            {\footnotesize Separación de contenido y estilo, programable, estándar,
            ligero, excelente calidad tipográfica, gestión automática de referencias,
            etc.}
      \item {\large Presentacion en PDF.}\\
            {\footnotesize Estándar, portable, etc.}
      \item {\large Estilos predefinidos elegantes y con herramientas útiles.}\\
            {\footnotesize Cabeceras y pies de página informativos, botones de 
            navegación, tablas de contenidos, etc.}
      \item {\large Fácil generación de overlays y efectos dinámicos.}
      \item {\large Software libre y gratuito, con una amplísima comunidad de soporte.}
    \end{itemize}
  \end{block}
\end{frame}

%%%%%%%%%%%%%%%%%%%%%%%%%%%%%%%%%%%%%%%%%%%%%%%%%%%%%%%%%%%%%%%%%%%%%%%%%%%%%%%%%%%%%%%%%

\section{Fórmulas matemáticas en Beamer}

%%%%%%%%%%%%%%%%%%%%%%%%%%%%%%%%%%%%%%%%%%%%%%%%%%%%%%%%%%%%%%%%%%%%%%%%%%%%%%%%%%%%%%%%%%

\begin{frame}
  \frametitle{Fórmulas matemáticas en Beamer}
  \begin{block}{}
    En Beamer, además de poder utilizar todo el potencial de \LaTeX{} en lo que 
    se refiere a fórmulas matemáticas, tenemos a nuestra disposición amplias 
    posibilidades para la animación de diapositivas.
  \end{block}
  \vspace{1 true cm}
  \begin{block}{}
   A continuación veremos algunos ejemplos de fórmulas matemáticas y animaciones 
   con Beamer.
  \end{block}
  
\end{frame}

%%%%%%%%%%%%%%%%%%%%%%%%%%%%%%%%%%%%%%%%%%%%%%%%%%%%%%%%%%%%%%%%%%%%%%%%%%%%%%%%%%

\begin{frame}
  \frametitle{Ejemplos}
  
  \begin{itemize}[<+->]
    \item $a + \left(\frac{b}{c}\right)= \frac{ac+b}{c}$
    \item $\sideset{_{a}^{b}}{_{c}^{d}}\prod$
    \item $\int_{x=0}^{\infty} x\,\text{e}^{-x^2}\text{d}x=\frac{1}{2},\quad\text{e}^{i\pi}+1=0$
    \item $2\sqrt{2}\,,\quad 2^2\sqrt{2-\sqrt{2}}\,,\quad 2^3\sqrt{2-\sqrt{2+\sqrt{2}}}\,,\;\ldots$
    \item $\sqrt 2 = 1+\frac{1}{2+\frac{1}{2+\frac{1}{\ddots}}}$
  \end{itemize}
\end{frame}
  
%%%%%%%%%%%%%%%%%%%%%%%%%%%%%%%%%%%%%%%%%%%%%%%%%%%%%%%%%%%%%%%%%%%%%%%%%%%%%%%%%
  
\begin{frame}
  \frametitle{Bibliografía}
  \bibliographystyle{plain}
  \bibliography{mibiblio}
  \nocite{*}
\end{frame}  

%%%%%%%%%%%%%%%%%%%%%%%%%%%%%%%%%%%%%%%%%%%%%%%%%%%%%%%%%%%%%%%%%%%%%%%%%%%%%%%%%

\end{document}









